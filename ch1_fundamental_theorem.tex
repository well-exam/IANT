\documentclass[11pt]{article}

\usepackage{amsthm}
\usepackage{amsmath}
\usepackage{amssymb}

\newtheorem{theorem}{Theorem}

\theoremstyle{definition}
\newtheorem*{defn}{Definition}
\newtheorem*{note}{Note}

\begin{document}

\begin{theorem}
  Divisibility has the following properties:

  \begin{enumerate}
  \item \makebox[40ex][l]{\(n \mid n\)} (reflexive property)
  \item \makebox[40ex][l]{\(d \mid n\) and \(n \mid m\) implies \(d \mid m\)} (transitive property)
  \item \makebox[40ex][l]{\(d \mid n\) and \(d \mid m\) implies \(d \mid (an+bm)\)} (linearity property)
  \item \makebox[40ex][l]{\(d \mid n\) implies \(ad \mid an\)} (multiplication property)
  \item \makebox[40ex][l]{\(ad \mid an\) and \(a \neq 0 \) implies \(d \mid n\)} (cancellation property)
  \item \makebox[40ex][l]{\(1 \mid n\)} (1 divides every integer)
  \item \makebox[40ex][l]{\(n \mid 0\)} (every integer divides zero)
  \item \makebox[40ex][l]{\(0 \mid n\) implies \(n=0\)} (zero divides only zero)
  \item \makebox[40ex][l]{\(d \mid n\) and \(n \neq 0\) implies \(| d | \leq | n |\)} (comparison property)
  \item \makebox[40ex][l]{\(d \mid n\) and \(n \mid d\) implies \(| d | = | n |\)}
    \item \makebox[40ex][l]{\(d \mid n\) and \(d \neq 0\) implies \((n/d) \mid n\).}
  \end{enumerate}
\end{theorem}

\begin{theorem}
  Given any two integers a and b, there is a common divisor d of a and b of the form
  \[
d = ax + by,
\]
where x and y are integers. Moreover, every common divisor of a and b divides this d.
\end{theorem}

\begin{proof}
  First we assume that \(a \geq 0\) and \(b \geq 0\). We use induction on \(n\), where
  \(n = a + b\). If \(n = 0\) then \(a = b = 0\) and we can take \(d = 0\) with \(x
  = y = 0\). Assume, then, that the theorem has been proved for \(0,1,2,\ldots,n-1\).
  By symmetry, we can assume \(a \geq b\). If \(b = 0\) take \(d = a, x = 1, y = 0\).
  If \(b \geq 1\) apply the theorem to \(a - b\) and \(b\). Since \((a - b) + b =
  a = n - b \leq n - 1\), the induction assumption is applicable and there is a common
  divisor \(d\) of \(a - b\) and \(b\) of the form \(d = (a-b)x + by\). This \(d\)
  also divides \((a - b) + b = a\) so \(d\) is a common divisor of \(a\) and \(b\)
  and we have \(d = ax + (y - x)b\), a linear combination of \(a\) and \(b\). To
  complete the proof we need to show that every common divisor divides \(d\). But
  a common divisor divides \(a\) and \(b\) and hence, by linearity, divides \(d\).

  If \(a < 0\) or \(b < 0\)(or both), we can apply the result just proved to
  \(| a |\) and \(| b |\). Then there is a common divisor \(d\) of
  \(| a |\) and \(| b |\) of the form
  \[
  d = | a | x + | b | y.
  \]
  If \(a < 0\), \(| a | x = -ax = a(-x)\). Similarly, if \(b < 0\), \(|
  b | y = b(-y)\). Hence \(d\) is again a linear combination of \(a\) and \(b\).
\end{proof}

\begin{theorem}
  Given integers a and b, there is one and only one number d with the following properties:
  \begin{enumerate}
  \item \makebox[30ex][l]{\(d \geq 0\)} (d is nonnegative)
  \item \makebox[30ex][l]{\(d \mid a\) and \(d \mid b\)} (d is a common divisor of a and b)
  \item \makebox[30ex][l]{\(e \mid a\) and \(e \mid b\) implies \(e \mid d\)} (every common divisor divides d).
  \end{enumerate}
\end{theorem}

\begin{proof}
  By Theorem 2 there is at least one \(d\) satisfying condition 2 and 3. Also,
  \(-d\) satisfies these conditions. But if \(d'\) satisfies 2 and 3, then
  \(d \mid d'\) and \(d' \mid d\), so \(| d | = | d' |\). Hence
  there is exactly one \(d \geq 0\) satisfying 2 and 3.
\end{proof}

\begin{defn}
  The number \(d\) of Theorem 3 is called the greatest common divisor (gcd) of \(a\) and \(b\) and is denoted by \((a,b)\) or by
  \(aDb\). If \((a,b) = 1\) then \(a\) and \(b\) are said to be relatively prime.
\end{defn}


\begin{theorem}
  The gcd has the following properties:
  \begin{enumerate}
  \item \((a,b) = (b,a)\)\\
    \makebox[25ex][l]{\(aDb = bDa\)} (commutative law)
  \item \((a,(b,c)) = ((a,b),c)\)\\
    \makebox[25ex][l]{\(aD(bDc)=(aDb)Dc\)} (associative law)
  \item \((ac,bc) = |c|(a,b)\)\\
    \makebox[25ex][l]{\((ca)D(cb) = |c|(aDb)\)} (distribute law)
  \item \((a,1) = (1,a) = 1, \qquad (a,0) = (0,a) = |a|\).\\
    \(aD1 = 1Da = 1, \qquad aD0 = 0Da = |a|\).
  \end{enumerate}
\end{theorem}

\begin{proof}
  Let \(d = (a, b)\) and let \(e = (ac, bc)\). We wish to prove that \(e = |c|d\). Write
  \(d = ax + by\). Then we have
  \begin{equation}
    cd = acx + bcy.
  \end{equation}
  Therefore \(cd \mid e\) because \(cd\) divides both \(ac\) and \(bc\). Also,  Equation (1) shows
  that \(e \mid cd\) because \(e \mid ac\) and \(e \mid bc\). Hence \(|e| = |cd|\), or \(e = |c|d\).
\end{proof}

\begin{theorem}[Euclid's lemma]
  If \(a \mid bc\) and if \((a,b) = 1\), then \(a \mid c\).
\end{theorem}

\begin{proof}
  Since \((a,b) = 1\) we can write \(1 = ax + by\). Therefore \(c = acx + bcy\).
  But \(a \mid acx\) and \(a \mid bcy\), so \(a \mid c\).
\end{proof}

\begin{defn}
  An interger \(n\) is called prime if \(n > 1\) and if the only positive
  divisors of \(n\) are \(1\) and \(n\). If \(n > 1\) and if \(n\) is not prime,
  then \(n\) is called composite.
\end{defn}

\begin{note}
  Prime numbers are usually denoted by \(p\), \(p'\), \(p_{i}\), \(q\), \(q'\), \(q_{i}\).
\end{note}

\begin{theorem}
  Every integer \(n > 1\) is either a prime number or a  product of prime numbers.
\end{theorem}

\begin{proof}
  We use induction on \(n\). The theorem is clearly true for \(n = 2\). Assume
  it is true for every integer \(<n\). Then if \(n\) is not prime it has a positive divisor
  \(d \neq 1\), \(d \neq n\). Hence \(n = cd\), where \(c \neq n\). But both \(c\) and \(d\)
  are \(<n\) and \(>1\) so each of \(c\), \(d\) is a product of prime numbers, hence so is n.
\end{proof}

\begin{theorem}[Euclid]
  There are infinitily many prime numbers.
\end{theorem}

\begin{proof}[Euclid's Proof]
  Suppose there are only a finite number, say \(p_{1}\), \(p_{2}\), \(\ldots\), \(p_{n}\).
  Let \(N = 1 + p_{1}p_{2}\cdots p_{n}\). Now \(N > 1\) so either \(N\) is prime or \(N\)
  is a product of primes. Of course \(N\) is not prime since it exceeds each \(p_{i}\).
  Moreover, no \(p_{i}\) divides \(N\) (if \(p_{i}\mid N\) then \(p_{i}\) divides the difference
  \(N - p_{1}p_{2}\cdots p_{n}=1\)). This contradicts Theorem 6.
\end{proof}

\begin{theorem}
  If a prime \(p\) does not divides \(a\), then \((p, a) = 1\).
\end{theorem}
\begin{proof}
  Let \(d = (p, a)\). Then \(d \mid p\) so \(d = 1\) or \(d = p\). But \(d \mid a\) so \(d \neq p\)
  because \(p \nmid a\). Hence d = 1.
\end{proof}

\begin{theorem}
  If a prime \(p\) divides \(ab\), the \(p \mid a\) or \(p \mid b\). More generally, if a
  prime \(p\) divides a product \(a_{1}\cdots a_{n}\), then \(p\) divides at least one of the factors.
\end{theorem}
\begin{proof}
  Assume \(p \mid ab\) and that \(p \nmid a\). We shall prove that \(p \mid b\). By Theorem 8,
  \((p, a) = 1\) so, by Euclid's lemma, \(p \mid b\).

  To prove the more general statement we use induction on \(n\), the number of factors. Details are left to the reader.
\end{proof}

\begin{theorem}[Fundamental theorem of arithmetic]
  Every integer \(n > 1\) can be represented as a product of prime factors in only one way, apart from the
  order of the factors.
\end{theorem}

\begin{proof}
  We use induction on \(n\). The theorem is true for \(n = 2\). Assume, then, that it is true for all
  integers greater than \(1\) and less than \(n\). We shall prove that \(n\) is composite and that \(n\) has
  two factorizations, say
  \begin{equation}
    \label{eq:2}
    n = p_{1}p_{2}\cdots p_{s} = q_{1}q_{2} \cdots q_{t}.
  \end{equation}
  We wish to show that \(s = t\) and that each \(p\) equals some \(q\). Since \(p_{1}\) divides the product
  \(q_{1}q_{2}\cdots q_{t}\) it must divides at least one factor. Relabel \(q_{1}\), \(q_{2}\), \(\cdots\), \(q_{t}\)
  so that \(p_{1} \mid q_{1}\). Then \(p_{1} = q_{1}\) since both \(p_{1}\) and \(q_{1}\) are primes. In~\ref{eq:2} we may
  cancel \(p_{1}\) on both sides to obtain
  \begin{equation*}
    n/p_{1} =  p_{2}\cdots p_{s} = q_{2} \cdots q_{t}.
  \end{equation*}
  If \(s > 1\) or \(t > 1\) then \(1 < n/p_{1} < n\). The induction hypothesus tells us that
  the two factorizations of \(n/p_{1}\) must be identical, apart from the order of the factors.
  Therefore \(s = t\) and the factorizations in~\ref{eq:2} are also identical, apart from order.
  This completes the proof.
\end{proof}

\begin{note}
  In the factorization of an integer \(n\), a particular prime \(p\) may occur more than once. If the \emph{distinct}
  prime factors of \(n\) are \(p_{1}\), \(\ldots\), \(p_{r}\) and if \(p_{i}\) occurs as a factor \(a_{i}\) times, we can write
  \begin{equation*}
    n = p_{1}^{a_{1}}\cdots p_{r}^{a_{r}}
  \end{equation*}
  or, more briefly,
  \begin{equation*}
    n = \prod_{i=1}^{r}p_{i}^{a_{i}}.
  \end{equation*}
  This is called the factorization of \(n\) into prime powers. We can also express \(1\)
  in this form by taking each exponent \(a_{i}\) to be \(0\).
\end{note}

\begin{theorem}
  If \(n = \prod_{i=1}^{r}p_{i}^{a_{i}}\), the set of positive divisors of \(n\) is the set of numbers of the form
  \(n = \prod_{i=1}^{r}p_{i}^{c_{i}}\), where \(0 \leq c_{i} \leq a_{i}\) for \(i = 1, 2, \ldots, r\).
\end{theorem}

\end{document}
